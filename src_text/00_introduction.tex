\begin{introduction}
Detection of the people and their subsequent identity preservation (tracking) in a video sequence is part of a more broad term called \gls{MOT}, which covers many different problems. In the basic definition, \gls{MOT} tries to estimate the position of the objects from multiple predefined classes and then preserve their identity through the whole sequence. Position estimation is done by object detector which predicts the bounding boxes with real-valued confidences of each object class in each video frame. However, an object detector means does not guarantee the relationship between objects in consecutive frames. Therefore, it is necessary to extract additional features from each object detected so the relationship in the sequence of frames can be built. The extracted features are mainly based on a visual appearance or movement of the object, but complementary information such as camera calibration and known scene parameters can also be incorporated. The subsequent tracking is then achieved by matching detected objects to preserved tracks based on various distance metrics that are calculated between features of the detected objects in a current frame and features of tracks from previous frames. One of the tracking benefits is that we can recover the trajectories of the objects that appeared in the video sequence. 

\section{Motivation}
    \gls{MOT} is one of the important and popular topics in computer vision field. For example, 99 \gls{MOT} tracking algorithms were submitted to the MOT17 challenge \cite{MOT16} during the year 2017 and similarly previous years. Since the accuracy of existing algorithms is increasing, new and harder datasets are being invented. VisDrone2018 \cite{zhuvisdrone2018} is a current state-of-the-art \gls{MOT} dataset with various object categories captured from drones. This demonstrates that \gls{MOT} problem, especially with objects like people, vehicles, bicycles, etc., has an enormous attention in the research community. This work will focus only on the MOT with people objects, however, many principles are the same in the more general variant with multiple classes.
    
    The logical extension beyond detection and tracking horizons is the extraction of additional class-specific features. If we would track cars, we could take advantage of estimation of car paint, brand, and type. This information can then be used for various temporal and regional statistics - for example, for estimating the richness of a town by counting luxury-type cars. If we take another case, which is the extraction of class-specific features about the people, we might be interested in the estimation of race, gender, height, mood, hair color, and clothes color. These traits are called soft biometrics, and they are mostly used in cases where we need to complement primary biometric identifiers, such as fingerprint, palm veins, iris pattern, to provide authentication based on the unique identification of the person.
    
    Although soft biometric characteristics lack the distinctiveness and permanence to recognize an individual uniquely and reliably, and can be easily faked, they provide some evidence about the people identity that could be beneficial. With the use of soft biometrics we can differentiate individuals in surveillance video where it is very common that people are often occluded. In other words, despite the fact they are unable to individualize a subject, they are effective in distinguishing between people, thus maintaining people's identity in surveillance scene. Another useful utilization is in the retail environment where we can build aggregated statistics such as number of woman between age 25 and 40 visited our store in the morning. If we have information that in the morning there is 75 \% women of visitors, we could utilize this and adapt the store to be more suitable for women, and therefore we will have a better chance to increase profit.
    
    \section{Challenges}
    \section{Thesis goals}
    \section{Thesis structure}
    The rest of this thesis is organized as follows. In the first chapter, we present existing methods which are crucial for understanding of this task. Chapter 2 is devoted to analysis with proposals how the thesis assignment can be solved. Design of the algorithms is done in chapter 3. Implementation details are explained in the chapter 4, followed by the evaluation presented in chapter 5. The latest is a summary of all the findings and suggestions for future improvements.
    
\end{introduction}
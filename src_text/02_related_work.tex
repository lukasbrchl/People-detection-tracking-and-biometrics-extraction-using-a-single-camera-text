\chapter{Related work}
In this chapter, we briefly present relevant work addressing similar problems \gls{mot}

\section{\Glsentryfull{mot}}
    Given the vastness of the topic, we only limit this section to significant work. Most state-of-the-art works follow the tracking-by-detection approach which heavily relies on the performance of the underlying detector method. However, the trend is now shifting to the end-to-end learning solutions and constructing stronger similarity scores based on appearance, motion, and interaction cues. Although recent \gls{nn} based detectors have outperformed all other methods for detection \cite{russakovsky2015imagenet, ren2015faster, redmon2016you}, it remains a challenging problem.
   
    \Gls{sort} \cite{bewley2016simple} is the first pragmatic approach where the main focus is to associate objects efficiently for online and real-time applications. They showed that the quality of detections plays a crucial role in tracking performance - according to their experiments, they can improve tracking by almost 20 \%, depending on the detector. Despite using an only simple combination of standard techniques as the Kalman Filter for motion prediction and Hungarian algorithm for the association of the tracks, they were able to achieve comparable performance to other state-of-the-art online trackers. 
    
    By adding a deep association metric to \gls{sort} \cite{wojke2017simple}, it was successfully integrated with the appearance model that improves the tracking performance. The appearance model is based on \gls{cnn} trained on large scale person \gls{reid} dataset. Due to this extension, the algorithm can track objects through longer periods of occlusion, effectively reducing the number of identity switches. The framework of this paper is used for the tracking task. Thus it is more explained in the next chapter. 

    In the 2016, revolutionary end-to-end learning approach based on \gls{rnn} has been introduced in a novel \cite{milan2017online}. Their proposed \gls{lstm} architecture is capable of performing all multi-target tracking tasks including prediction, data association, state update as well as initiation and termination of targets within a unified network structure. One of the main advantages of this approach is that it is completely model-free, i.e., it does not require any prior knowledge about target dynamics, clutter distributions, etc. However, the detections should be given as input.
    
    It was not the only case of successful use of \gls{rnn}. In the paper published in 2017 \cite{sadeghian2017tracking}, a new approach combining multiple cues such as appearance, movement, and interaction are utilized in \gls{lstm} architecture, which learns and remembers the dependencies in a sequence of observation, in contrast to pairwise similarity where only the observations from the current and previous frames are used. Their proposed framework follows end-to-end fashion.
    
    Competitive tracking results can be achieved even without sophisticated tracking methods. Tractor \cite{DBLP:journals/corr/abs-1903-05625} accomplishes tracking without following the tracking-by-detection approach - authors of this work performed no training or optimization on tracking data. They exploited the bounding box regression of an object detector to perform temporal realignment and to predict the position of an object in the next frame. They also provide a simple extension to this approach, in the form of Siamese \gls{nn} for \gls{reid} and motion analysis model, which achieve state-of-the-art performance on tracking benchmarks. 
    
    Results of recent tracking evaluations show that bounding box level tracking performance is saturating \cite{mot16}. Further improvements will only be possible when moving to the pixel level. This is why authors of recent work from Feb 2019 are expanding from \gls{mot} to \gls{mots}. They propose new TrackR-CNN baseline method which jointly addresses detection, tracking, and segmentation with a single convolutional network that extends Mask R-CNN architecture with 3D convolutions to incorporate temporal information and by an association head which is used to link object identities over time. They also provide evaluation metrics and new dataset with masks for over one thousand distinct objects in ten thousand frames. The main advantage of \gls{mots} is that segmentation based tracking results, are by definition non-overlapping and can thus be compared to ground truth straightforwardly.
    
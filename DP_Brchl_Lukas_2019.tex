% arara: xelatex
% arara: xelatex
% arara: xelatex

% options:
% thesis=B bachelor's thesis
% thesis=M master's thesis
% czech thesis in Czech language
% english thesis in English language
% hidelinks remove colour boxes around hyperlinks

\documentclass[thesis=M,english]{prefs/FITthesis}[2019/03/06]

% \usepackage{subfig} %subfigures
% \usepackage{amsmath} %advanced maths
% \usepackage{amssymb} %additional math symbols

\usepackage[utf8]{inputenc}

\usepackage{graphicx} %graphics files inclusion
% \usepackage{amsmath} %advanced maths
% \usepackage{amssymb} %additional math symbols

\usepackage{dirtree} %directory tree visualisation
\usepackage{subfig} %image side by side
\usepackage{todonotes} %todo
\usepackage{url}
\usepackage{textcomp} %degree symbol
\usepackage{color, colortbl} %color, table color
\usepackage{enumitem} %lists
\usepackage{float} %for H option in figures
\usepackage{array} %table aligment       
\usepackage{amsmath} %cases
\usepackage{svg} %svg
\usepackage{scrextend}
\usepackage{multirow}
\addtokomafont{labelinglabel}{\sffamily}


% list of acronyms
\usepackage[acronym,nonumberlist,toc,numberedsection=autolabel,nomain]{glossaries}
\iflanguage{czech}{\renewcommand*{\acronymname}{Seznam pou{\v z}it{\' y}ch zkratek}}{}
\makeglossaries

\newcommand{\tg}{\mathop{\mathrm{tg}}} %cesky tangens
\newcommand{\cotg}{\mathop{\mathrm{cotg}}} %cesky cotangens

% % % % % % % % % % % % % % % % % % % % % % % % % % % % % % % % % % % 
% % % % % % % % % % % % % % % % % % % % % % % % % % % % % % % % % % % 
\department{Department of Applied Mathematics}
\title{People detection, tracking and biometric data extraction using a single camera for retail
usage}
\authorGN{Luk{\' a}{\v s}} %author's given name/names
\authorFN{Brchl} %author's surname
\authorWithDegrees{Bc. Luk{\' a}{\v s} Brchl} %author's name with academic degrees
\author{Luk{\' a}{\v s} Brchl} %author's name without academic degrees
\supervisor{doc. RNDr. Ing. Marcel Jiřina, Ph.D.}
\acknowledgements{}
\abstractCS{}
\abstractEN{This thesis aims to design and implement a framework that analyzes video sequences to extract as much information as possible about the people in the scene captured by a single RGB camera. The whole framework can be broken down into the smaller components, i.e. people detector, people tracker. and biometrics extractor. The people detector employs a well-known deep-learning architecture to estimate bounding boxes of individuals. The tracking solution is built to be robust regarding the crowded scenes by incorporating multiple features in matching phase such as person's visual appearance, motion, face, and height. Each new detection has these features calculated by various computer vision algorithms and then these hypotheses are matched against existing tracks utilizing multiple distance metrics. Apart from using the features only for matching, they are kept for calculation of various output statistics. The approach is validated against the dataset which was created for this propose. The integration of face information in matching phase greatly improves the performance, however it is not always possible to extract it. This face information also helps to maintain a person identity even when when they leave the scene and then reappears. At the end, it is shown what the output statistics can be used for.}
\placeForDeclarationOfAuthenticity{} %where you have signed the declaration
\keywordsCS{počítačové vidění, detekce osob, sledování osob, extrakce biometrických údajů}
\keywordsEN{computer vision, people detection, people tracking, biometrics extraction}
\declarationOfAuthenticityOption{5} %select as appropriate, according to the desired license
\website{https://github.com/lukasbrchl/People-detection-tracking-and-biometrics-extraction-using-a-single-camera-text} %optional URL (remove entirely if you have no URL for this thesis)

\begin{document}
\newacronym{MOT}{MOT}{multiple object tracking}

\begin{introduction}
    \section{Motivation}
    \section{Challenges}
    \section{Thesis goals}
    \section{Document structure}
\end{introduction}
% theoretical background
\chapter{Existing methods}
In this chapter, we provide a theoretical background of 
In this chapter I will provide you a brief theoretical background. I will tell
what computer vision is and how it’s related to the main problem of this
thesis. I will discuss several types of computer vision algorithms. Some of
them were used to implement application (see Chapter 4). At the end of the
chapter, problem of occlusion is also mentioned.
\input{src_text/02_analysis.tex}
\input{src_text/03_design.tex}
\chapter{Implementation}

% Debugging tools

%It contains a function
% cvFindHomography() that takes in a set of point correspondences and returns a homography matrix H. This function makes use of the normalized
% DLT algorithm discussed in section 2.1.1 to estimate H.

\chapter{Experiments and evaluation}

\section{Dataset}
    There are several public annotated datasets \cite{ferryman2009pets2009, MOTChallenge2015, zhuvisdrone2018} for a person tracking problem. However, their focus is a little different than what is needed for evaluation of this thesis. Firstly,  
    individuals only appear in a few images of the sequence, which is not entirely the case from retail, where it is required to maintain the individual's long-term identity even in the case of multiple occlusions. Secondly, these datasets lack of additional information about people and the scene parameters that are required to calibrate the camera. It was therefore advisable to create a dataset to test these cases.
    
    The 14th floor of the FSv CTU building was used to create a dataset, where it was possible to place a PC and a camera. In total, over 10,000 images were captured, of which 2463 frames were hand-annotated with bounding-boxes and person identities. There are a total of 3265 bounding boxes with 11 person identities. Captured frames are not further pre-processed.
  
\section{Results}
    In order to be able to compare individual solutions among each other, it is necessary to determine a suitable metric. Due to the fact that metrics for object detection and tracking in the video can often be non-intuitive and complex, the most well-known MOTP and MOTA metrics are adapted in this work. MOTD metrics focus on the quality of detected regions, while MOTA focuses on tracking accuracy, and its calculation affects the number of false positives, undetected watches, and identity swapping. Factors affecting MOTA are also compared, as it is interesting to see how they change depending on the detection network. The metrics are described in more detail in section \ref{evaluation-metrics}, however, to be concise, higher MOTA or MOTP means better.
    

%Things that didn't work well

%Further works
% Although KF is designed based on a criterion of linear minimum square errors, the optimal estimation
% can be achieved by Kalman filtering only if the following three assumptions are met simultaneously: (1)
% the state model and measurement model are both linear; (2) the model is designed well to fit the actual
% system; (3) the noise is additive Gaussian noise. In actual or real systems, the second and third
% assumptions are quite difficult to be met. The designed model will deviate, more or less, from the
% actual system. 

%child, youth, adult, middleage and elderly
\begin{conclusion}

This work has provided an extensive overview of topics employed in \gls{cv} applications today.
Based on that some popular methods were implemented from scratch or taken over from open-source repositories and customized to meet the thesis assignment.

The proposed methods are designed in several variations to meet different trade-offs of accuracy versus computational complexity. and since the final design of the framework is fully modular, it can be easily configured to meet demands for various other scenarios. Quite a lot of work has been done regarding design of the framework as a whole so that it can be easily expanded and it can now be deployed in real-world applications.

There are still problems that have not yet been addressed, especially with regards to 

To conclude, this thesis has provided 
and we hope that this work will be a useful as a staring point for other people interested in this topic. The result is a functioning people tracking and soft-biometrics extracting framework.

\end{conclusion}

% bibliography
\bibliographystyle{prefs/iso690}
\bibliography{ref}

\appendix

% acronyms
\printglossaries

% media contents
\chapter{Media contents}\label{app:CDcontent}
\begin{figure}
% 	\dirtree{%
% 		.1 readme.txt\DTcomment{the file with CD contents description}.
% 		.1 data\DTcomment{the data files directory}.
% 		.2 graphs\DTcomment{the directory of graphs of experiments}.
% 		.3 *.eps\DTcomment{the B/W graphs}.
% 		.3 *.png\DTcomment{the color graphs}.
% 		.3 *.dat\DTcomment{the graphs data files}.
% 		.1 exe\DTcomment{the directory with executable WBDCM program}.
% 		.2 wbdcm\DTcomment{the WBDCM program executable (UNIX)}.
% 		.2 wbdcm.exe\DTcomment{the WBDCM program executable (Windows)}.
% 		.1 src\DTcomment{the directory of source codes}.
% 		.2 wbdcm\DTcomment{the directory of WBDCM program}.
% 		.3 Makefile\DTcomment{the makefile of WBDCM program (UNIX)}.
% 		.2 thesis\DTcomment{the directory of \LaTeX{} source codes of the thesis}.
% 		.3 figures\DTcomment{the thesis figures directory}.
% 		.3 *.tex\DTcomment{the \LaTeX{} source code files of the thesis}.
% 		.1 text\DTcomment{the thesis text directory}.
% 		.2 thesis.pdf\DTcomment{the Diploma thesis in PDF format}.
% 		.2 thesis.ps\DTcomment{the Diploma thesis in PS format}.
% 	}
\end{figure}


\end{document}
